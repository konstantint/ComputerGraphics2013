\documentclass{article}
\usepackage{amsmath, amssymb, mdwlist, graphicx, hyperref}
\usepackage{listings,color}
\usepackage{wrapfig}
\usepackage[usenames,dvipsnames]{xcolor}
\definecolor{gray}{rgb}{0.97,0.97,0.97}
\lstset{%
language=C,%
%backgroundcolor=\color{gray},
emph={putpixel},
emphstyle=\bf,
tabsize=4,
framesep=5pt,
mathescape=true,
xleftmargin=0.1cm,
xrightmargin=0.1cm,
frame=lines,
%basicstyle=\ttfamily,
%keywordstyle=\color{Blue},
%commentstyle=\color{OliveGreen},
%stringstyle=\color{MidnightBlue},
columns=flexible,
%showstringspaces=false
}

\newcommand{\mpar}[1]{\marginpar{\textit{#1}}}
\newcommand{\norm}[1]{\Vert #1 \Vert}
\DeclareMathOperator{\argmax}{argmax}
\DeclareMathOperator{\argmin}{argmin}
\newenvironment{solution}{\paragraph{Solution.}$\,$ }{\vskip 3mm\hrule}
\newenvironment{exercise}[2]{\paragraph{Exercise #1 (#2pt).} }{
\medskip}
\newcommand{\bbR}{\mathbb{R}}
\newcommand{\bw}{\mathbf{w}}
\newcommand{\bx}{\mathbf{x}}
\newcommand{\bd}{\mathbf{d}}
\newcommand{\bb}{\mathbf{b}}
\newcommand{\by}{\mathbf{y}}
\newcommand{\bzero}{\mathbf{0}}
\newcommand{\bz}{\mathbf{z}}
\newcommand{\bSigma}{\mathbf{\Sigma}}
\newcommand{\bp}{\mathbf{p}}
\newcommand{\bP}{\mathbf{P}}
\newcommand{\bm}{\mathbf{m}}
\newcommand{\bc}{\mathbf{c}}
\newcommand{\bM}{\mathbf{M}}
\newcommand{\bV}{\mathbf{V}}
\newcommand{\bK}{\mathbf{K}}
\newcommand{\bD}{\mathbf{D}}
\newcommand{\bA}{\mathbf{A}}
\newcommand{\bX}{\mathbf{X}}
\newcommand{\bY}{\mathbf{Y}}
\newcommand{\bR}{\mathbf{R}}
\newcommand{\bI}{\mathbf{I}}
\newcommand{\bS}{\mathbf{S}}
\newcommand{\bT}{\mathbf{T}}
\newcommand{\balpha}{\boldsymbol{\alpha}}
\newcommand{\pt}[2]{\left(\begin{array}{c}#1\\#2\end{array}\right)}

\begin{document}
\title{MTAT.03.015 Computer Graphics (Fall 2013)\\
Exercise session XIV: OGRE.}
\author{Konstantin Tretyakov, Ilya Kuzovkin}
\date{December 9, 2013}
\maketitle

In this exercise session we will have a look at higher-level rendering engine called OGRE\footnote{\url{http://www.ogre3d.org/}}. We will compare it to the things we've learned so far and see how the concepts we know about are included in the OGRE engine. Lighting, materials, shadows, environmental mapping and other techniques are made accessible to us without the need to implement all annoying details on our own, but rather by adding few lines of code.

The solutions will have to be submitted as a zipped project directory. Please do not remove Windows libraries if you work on Linux and vice versa.

\section{Structure of the Application}
We start by comparing the structure of the application to the familiar structure we've been using so far. Please open \verb#1_OgreTriangle# project and read through the code in the \verb#main.cpp# file. Compare it to the GLUT-based applications we have seen before.

Triangle: Add another 0.5

Lighting: Light, Material 0.5, Stencil shadows* 0.5

\section{High-levelness}

TODO: mesh, shadow mapping 0.5, environmental mapping 1, textures* 1 

\section{Plugins}

TODO: Particle system, Some other funny plugin to demonstrate scene change

\end{document}
